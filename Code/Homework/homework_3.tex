% Options for packages loaded elsewhere
\PassOptionsToPackage{unicode}{hyperref}
\PassOptionsToPackage{hyphens}{url}
%
\documentclass[
]{article}
\title{Homework Week 4: Population counts for endangered species}
\author{}
\date{\vspace{-2.5em}}

\usepackage{amsmath,amssymb}
\usepackage{lmodern}
\usepackage{iftex}
\ifPDFTeX
  \usepackage[T1]{fontenc}
  \usepackage[utf8]{inputenc}
  \usepackage{textcomp} % provide euro and other symbols
\else % if luatex or xetex
  \usepackage{unicode-math}
  \defaultfontfeatures{Scale=MatchLowercase}
  \defaultfontfeatures[\rmfamily]{Ligatures=TeX,Scale=1}
\fi
% Use upquote if available, for straight quotes in verbatim environments
\IfFileExists{upquote.sty}{\usepackage{upquote}}{}
\IfFileExists{microtype.sty}{% use microtype if available
  \usepackage[]{microtype}
  \UseMicrotypeSet[protrusion]{basicmath} % disable protrusion for tt fonts
}{}
\makeatletter
\@ifundefined{KOMAClassName}{% if non-KOMA class
  \IfFileExists{parskip.sty}{%
    \usepackage{parskip}
  }{% else
    \setlength{\parindent}{0pt}
    \setlength{\parskip}{6pt plus 2pt minus 1pt}}
}{% if KOMA class
  \KOMAoptions{parskip=half}}
\makeatother
\usepackage{xcolor}
\IfFileExists{xurl.sty}{\usepackage{xurl}}{} % add URL line breaks if available
\IfFileExists{bookmark.sty}{\usepackage{bookmark}}{\usepackage{hyperref}}
\hypersetup{
  pdftitle={Homework Week 4: Population counts for endangered species},
  hidelinks,
  pdfcreator={LaTeX via pandoc}}
\urlstyle{same} % disable monospaced font for URLs
\usepackage[margin=1in]{geometry}
\usepackage{color}
\usepackage{fancyvrb}
\newcommand{\VerbBar}{|}
\newcommand{\VERB}{\Verb[commandchars=\\\{\}]}
\DefineVerbatimEnvironment{Highlighting}{Verbatim}{commandchars=\\\{\}}
% Add ',fontsize=\small' for more characters per line
\usepackage{framed}
\definecolor{shadecolor}{RGB}{248,248,248}
\newenvironment{Shaded}{\begin{snugshade}}{\end{snugshade}}
\newcommand{\AlertTok}[1]{\textcolor[rgb]{0.94,0.16,0.16}{#1}}
\newcommand{\AnnotationTok}[1]{\textcolor[rgb]{0.56,0.35,0.01}{\textbf{\textit{#1}}}}
\newcommand{\AttributeTok}[1]{\textcolor[rgb]{0.77,0.63,0.00}{#1}}
\newcommand{\BaseNTok}[1]{\textcolor[rgb]{0.00,0.00,0.81}{#1}}
\newcommand{\BuiltInTok}[1]{#1}
\newcommand{\CharTok}[1]{\textcolor[rgb]{0.31,0.60,0.02}{#1}}
\newcommand{\CommentTok}[1]{\textcolor[rgb]{0.56,0.35,0.01}{\textit{#1}}}
\newcommand{\CommentVarTok}[1]{\textcolor[rgb]{0.56,0.35,0.01}{\textbf{\textit{#1}}}}
\newcommand{\ConstantTok}[1]{\textcolor[rgb]{0.00,0.00,0.00}{#1}}
\newcommand{\ControlFlowTok}[1]{\textcolor[rgb]{0.13,0.29,0.53}{\textbf{#1}}}
\newcommand{\DataTypeTok}[1]{\textcolor[rgb]{0.13,0.29,0.53}{#1}}
\newcommand{\DecValTok}[1]{\textcolor[rgb]{0.00,0.00,0.81}{#1}}
\newcommand{\DocumentationTok}[1]{\textcolor[rgb]{0.56,0.35,0.01}{\textbf{\textit{#1}}}}
\newcommand{\ErrorTok}[1]{\textcolor[rgb]{0.64,0.00,0.00}{\textbf{#1}}}
\newcommand{\ExtensionTok}[1]{#1}
\newcommand{\FloatTok}[1]{\textcolor[rgb]{0.00,0.00,0.81}{#1}}
\newcommand{\FunctionTok}[1]{\textcolor[rgb]{0.00,0.00,0.00}{#1}}
\newcommand{\ImportTok}[1]{#1}
\newcommand{\InformationTok}[1]{\textcolor[rgb]{0.56,0.35,0.01}{\textbf{\textit{#1}}}}
\newcommand{\KeywordTok}[1]{\textcolor[rgb]{0.13,0.29,0.53}{\textbf{#1}}}
\newcommand{\NormalTok}[1]{#1}
\newcommand{\OperatorTok}[1]{\textcolor[rgb]{0.81,0.36,0.00}{\textbf{#1}}}
\newcommand{\OtherTok}[1]{\textcolor[rgb]{0.56,0.35,0.01}{#1}}
\newcommand{\PreprocessorTok}[1]{\textcolor[rgb]{0.56,0.35,0.01}{\textit{#1}}}
\newcommand{\RegionMarkerTok}[1]{#1}
\newcommand{\SpecialCharTok}[1]{\textcolor[rgb]{0.00,0.00,0.00}{#1}}
\newcommand{\SpecialStringTok}[1]{\textcolor[rgb]{0.31,0.60,0.02}{#1}}
\newcommand{\StringTok}[1]{\textcolor[rgb]{0.31,0.60,0.02}{#1}}
\newcommand{\VariableTok}[1]{\textcolor[rgb]{0.00,0.00,0.00}{#1}}
\newcommand{\VerbatimStringTok}[1]{\textcolor[rgb]{0.31,0.60,0.02}{#1}}
\newcommand{\WarningTok}[1]{\textcolor[rgb]{0.56,0.35,0.01}{\textbf{\textit{#1}}}}
\usepackage{graphicx}
\makeatletter
\def\maxwidth{\ifdim\Gin@nat@width>\linewidth\linewidth\else\Gin@nat@width\fi}
\def\maxheight{\ifdim\Gin@nat@height>\textheight\textheight\else\Gin@nat@height\fi}
\makeatother
% Scale images if necessary, so that they will not overflow the page
% margins by default, and it is still possible to overwrite the defaults
% using explicit options in \includegraphics[width, height, ...]{}
\setkeys{Gin}{width=\maxwidth,height=\maxheight,keepaspectratio}
% Set default figure placement to htbp
\makeatletter
\def\fps@figure{htbp}
\makeatother
\setlength{\emergencystretch}{3em} % prevent overfull lines
\providecommand{\tightlist}{%
  \setlength{\itemsep}{0pt}\setlength{\parskip}{0pt}}
\setcounter{secnumdepth}{-\maxdimen} % remove section numbering
\ifLuaTeX
  \usepackage{selnolig}  % disable illegal ligatures
\fi

\begin{document}
\maketitle

First, we install the required packages: devtools, tidyverse and vroom.

We then import both datasets using vroom:

\begin{Shaded}
\begin{Highlighting}[]
\NormalTok{pop\_1 }\OtherTok{\textless{}{-}} \FunctionTok{vroom}\NormalTok{(}\StringTok{"https://raw.githubusercontent.com/chrit88/Bioinformatics\_data/master/Workshop\%203/to\_sort\_pop\_1.csv"}\NormalTok{)}
\end{Highlighting}
\end{Shaded}

\begin{verbatim}
## Rows: 30 Columns: 29
\end{verbatim}

\begin{verbatim}
## -- Column specification --------------------------------------------------------
## Delimiter: "\t"
## chr  (4): species, primary_threat, secondary_threat, tertiary_threat
## dbl (24): pop_1_2003-01-01, pop_1_2004-01-01, pop_1_2005-01-01, pop_1_2006-0...
## lgl  (1): pop_1_1995-01-01
\end{verbatim}

\begin{verbatim}
## 
## i Use `spec()` to retrieve the full column specification for this data.
## i Specify the column types or set `show_col_types = FALSE` to quiet this message.
\end{verbatim}

\begin{Shaded}
\begin{Highlighting}[]
\FunctionTok{head}\NormalTok{(pop\_1)}
\end{Highlighting}
\end{Shaded}

\begin{verbatim}
## # A tibble: 6 x 29
##   species      primary_threat  secondary_threat tertiary_threat `pop_1_2003-01-~
##   <chr>        <chr>           <chr>            <chr>                      <dbl>
## 1 Schistidium~ Habitat destru~ <NA>             <NA>                          NA
## 2 Paraleucobr~ Exploitation    Habitat loss     <NA>                          NA
## 3 Scapania pa~ Climate change  <NA>             <NA>                          NA
## 4 Seligera re~ Exploitation    <NA>             <NA>                          NA
## 5 Tortula sub~ Habitat loss    Pollution        Climate change                96
## 6 Pohlia mela~ <NA>            <NA>             <NA>                         288
## # ... with 24 more variables: pop_1_2004-01-01 <dbl>, pop_1_2005-01-01 <dbl>,
## #   pop_1_2006-01-01 <dbl>, pop_1_2007-01-01 <dbl>, pop_1_2008-01-01 <dbl>,
## #   pop_1_2009-01-01 <dbl>, pop_1_2010-01-01 <dbl>, pop_1_2011-01-01 <dbl>,
## #   pop_1_2012-01-01 <dbl>, pop_1_2013-01-01 <dbl>, pop_1_2014-01-01 <dbl>,
## #   pop_1_2015-01-01 <dbl>, pop_1_2016-01-01 <dbl>, pop_1_2017-01-01 <dbl>,
## #   pop_1_2018-01-01 <dbl>, pop_1_2019-01-01 <dbl>, pop_1_2000-01-01 <dbl>,
## #   pop_1_2001-01-01 <dbl>, pop_1_2002-01-01 <dbl>, pop_1_1997-01-01 <dbl>, ...
\end{verbatim}

\begin{Shaded}
\begin{Highlighting}[]
\NormalTok{pop\_2 }\OtherTok{\textless{}{-}} \FunctionTok{vroom}\NormalTok{(}\StringTok{"https://raw.githubusercontent.com/chrit88/Bioinformatics\_data/master/Workshop\%203/to\_sort\_pop\_2.csv"}\NormalTok{)}
\end{Highlighting}
\end{Shaded}

\begin{verbatim}
## Rows: 30 Columns: 28
\end{verbatim}

\begin{verbatim}
## -- Column specification --------------------------------------------------------
## Delimiter: "\t"
## chr  (4): species, primary_threat, secondary_threat, tertiary_threat
## dbl (21): pop_2_2000-01-01, pop_2_2001-01-01, pop_2_2002-01-01, pop_2_2003-0...
## lgl  (3): pop_2_1996-01-01, pop_2_1997-01-01, pop_2_1998-01-01
\end{verbatim}

\begin{verbatim}
## 
## i Use `spec()` to retrieve the full column specification for this data.
## i Specify the column types or set `show_col_types = FALSE` to quiet this message.
\end{verbatim}

\begin{Shaded}
\begin{Highlighting}[]
\FunctionTok{head}\NormalTok{(pop\_2)}
\end{Highlighting}
\end{Shaded}

\begin{verbatim}
## # A tibble: 6 x 28
##   species       primary_threat secondary_threat tertiary_threat `pop_2_2000-01-~
##   <chr>         <chr>          <chr>            <chr>                      <dbl>
## 1 Sphagnum pal~ <NA>           <NA>             <NA>                          NA
## 2 Pohlia wahle~ Habitat loss   Pollution        <NA>                          NA
## 3 Sphagnum les~ Pollution      Exploitation     <NA>                          NA
## 4 Marchantia p~ <NA>           <NA>             <NA>                          NA
## 5 Platyhypnidi~ Habitat loss   <NA>             <NA>                          NA
## 6 Scleropodium~ Pollution      <NA>             <NA>                          NA
## # ... with 23 more variables: pop_2_2001-01-01 <dbl>, pop_2_2002-01-01 <dbl>,
## #   pop_2_2003-01-01 <dbl>, pop_2_2004-01-01 <dbl>, pop_2_2005-01-01 <dbl>,
## #   pop_2_2006-01-01 <dbl>, pop_2_2007-01-01 <dbl>, pop_2_2008-01-01 <dbl>,
## #   pop_2_2009-01-01 <dbl>, pop_2_2010-01-01 <dbl>, pop_2_2011-01-01 <dbl>,
## #   pop_2_2012-01-01 <dbl>, pop_2_2013-01-01 <dbl>, pop_2_2014-01-01 <dbl>,
## #   pop_2_2015-01-01 <dbl>, pop_2_2016-01-01 <dbl>, pop_2_2017-01-01 <dbl>,
## #   pop_2_2018-01-01 <dbl>, pop_2_2019-01-01 <dbl>, pop_2_1996-01-01 <lgl>, ...
\end{verbatim}

Now, we first add a new column to both population datasets to specify
which population they are (1 or 2)

\begin{Shaded}
\begin{Highlighting}[]
\NormalTok{pop\_1}\SpecialCharTok{$}\NormalTok{Population}\OtherTok{=}\DecValTok{1}
\NormalTok{pop\_2}\SpecialCharTok{$}\NormalTok{Population }\OtherTok{=}\DecValTok{2}
\end{Highlighting}
\end{Shaded}

Then, we can go ahead and use the ``merge'' function to merge the two
data frames, setting the all parameter to TRUE thus including all
columns in both data frames.

\begin{Shaded}
\begin{Highlighting}[]
\NormalTok{pops }\OtherTok{\textless{}{-}} \FunctionTok{merge}\NormalTok{(pop\_1,pop\_2, }\AttributeTok{all =} \ConstantTok{TRUE}\NormalTok{)}
\end{Highlighting}
\end{Shaded}

Finally, we convert the new merged data frame to long format, specifying
the 5 columns we want to keep and merging the other columns with new
variables ``Date'' and ``Pop count''.

\begin{Shaded}
\begin{Highlighting}[]
\NormalTok{pops\_long }\OtherTok{\textless{}{-}}\NormalTok{ pops }\SpecialCharTok{\%\textgreater{}\%}
  
  \FunctionTok{pivot\_longer}\NormalTok{(}\AttributeTok{cols =} \SpecialCharTok{{-}}\FunctionTok{c}\NormalTok{(species,}
\NormalTok{                         primary\_threat,}
\NormalTok{                         secondary\_threat,}
\NormalTok{                         tertiary\_threat,}
\NormalTok{                         Population),}
               \AttributeTok{names\_to =} \StringTok{"Date"}\NormalTok{,}
               \AttributeTok{values\_to =} \StringTok{"Pop count"}\NormalTok{)}

\FunctionTok{head}\NormalTok{(pops\_long)}
\end{Highlighting}
\end{Shaded}

\begin{verbatim}
## # A tibble: 6 x 7
##   species    primary_threat secondary_threat tertiary_threat Population Date    
##   <chr>      <chr>          <chr>            <chr>                <dbl> <chr>   
## 1 Acaulon t~ Pollution      Exploitation     Climate change           2 pop_1_2~
## 2 Acaulon t~ Pollution      Exploitation     Climate change           2 pop_1_2~
## 3 Acaulon t~ Pollution      Exploitation     Climate change           2 pop_1_2~
## 4 Acaulon t~ Pollution      Exploitation     Climate change           2 pop_1_2~
## 5 Acaulon t~ Pollution      Exploitation     Climate change           2 pop_1_2~
## 6 Acaulon t~ Pollution      Exploitation     Climate change           2 pop_1_2~
## # ... with 1 more variable: Pop count <dbl>
\end{verbatim}

\end{document}
